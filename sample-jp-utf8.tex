%%
% このファイルは、筑波大学情報学群情報メディア創成学類の
% 卒業研究論文本体のサンプルです。
% このファイルを書き換えて、この例と同じような書式の論文本体を
% LaTeXを使って作成することができます。
% 
% PC環境や、LaTeX環境の設定によっては漢字コードや改行コードを
% 変更する必要があります。
%%
\documentclass[a4paper,11pt]{jreport}

%%【PostScript, JPEG, PNG等の画像の貼り込み】
%% 利用するパッケージを選んでコメントアウトしてください。
%%
%% 推奨: graphicx パッケージ(本ファイルではこれを使用)
%%	動かない場合にいは \usepackage[dvipdfmx]{graphicx} のように dvi 変換コマンドを明示指定する。
%%
\usepackage{graphicx} % for \includegraphics[width=3cm]{sample.eps}
%% 一応 OK (epsfig.sy)
% \usepackage{epsfig} % for \psfig{file=sample.eps,width=3cm}
%% 以下の2つの使用はあまり勧められない (epsf.sty, epsfbox.sty)
%\usepackage{epsf} % for \epsfile{file=sample.eps,scale=0.6}
%\usepackage{epsbox} % for \epsfile{file=sample.eps,scale=0.6}

\usepackage{times} % use Times Font instead of Computer Modern

\setcounter{tocdepth}{3}	% 目次を3レベル (1.2.3) まで
\setcounter{secnumdepth}{3}	% 番号付けレベル。 3: \section まで 4: \subsubsection まで
\setcounter{page}{-1}

\setlength{\oddsidemargin}{0.1in}
\setlength{\evensidemargin}{0.1in} 
\setlength{\topmargin}{0in}
\setlength{\textwidth}{6in} 
%\setlength{\textheight}{10.1in}	% ページの縦幅を変更する場合には設定する
\setlength{\parskip}{0em}
\setlength{\topsep}{0em}

%% タイトル生成用パッケージ(重要)
\usepackage{mast-jp-utf8}

%% タイトル
%% 【注意】タイトルの最後に\\ を入れるとエラーになります
\title{大規模言語モデルを用いた語用論的\\アプローチに基づく誤解可能性を考慮した\\画像情報ラベリング}
%% 著者
\author{清野 駿}
%% 指導教員
\advisor{若林 啓}

%% 年月 (提出年月)
%% 年月は必要に応じて書き替えてください。
\majorfield{ } \yearandmonth{2024年 2月}



\begin{document}
\maketitle
\thispagestyle{empty}
\newpage

\thispagestyle{empty}
\vspace*{20pt plus 1fil}
\parindent=1zw
\noindent
%%
%% 論文の概要(Abstract)
%%
\begin{center}
{\bf 概要}
\vspace{5mm}
\end{center}
本研究では,複数の画像情報が与えられたときに,そのうちの一つを誤解なく指し示すことのできる自然言語ラベルを導出する手法を提案する.

% 背景
近年,画像情報に対して適切な自然言語ラベルを付与することのできる,マルチモーダルな学習済み大規模言語モデルが広く利用可能になっている.

% 課題
しかし,大量の画像情報を一度に処理して,それぞれに特有の特徴を反映した自然言語ラベルを付与することは容易ではない.

% 研究目的
本研究では,語用論に基づく「聞き手モデル」を導入し,内省的処理を通じてラベルの品質を向上させるアプローチを提案する.

% 手法
% 大規模言語モデルを素直に用いてラベルを生成する手法をベースライン手法とし,「聞き手モデル」を導入した「強化ベースライン手法」と,単語の組み合わせによるラベル生成・改善アルゴリズムを導入した「組合せ最適化手法」の2つの提案手法によって,それぞれラベルの品質を向上させる.

% 実験
実験ではベースライン手法と2つの提案手法の計3つの手法によってラベルを生成し,これらの手法によるラベルの誤解可能性を被験者実験を通じて比較した.
%%%%%
\par
\vspace{0pt plus 1fil}
\newpage

\pagenumbering{roman} % I, II, III, IV 
\tableofcontents
\listoffigures
%\listoftables		% 本ファイルでは省略してある

%% ここから本文
\pagebreak \setcounter{page}{1}
\pagenumbering{arabic} % 1,2,3


\chapter{はじめに}

画像情報に対して自然言語のテキストを生成する技術は,様々な産業的な応用への活用が期待される \cite{Yin2023}.
特に,複数の画像が与えられたときに,それらを誤解なく指し示すことのできるテキスト表現を生成することは,ロボットによる場所のナビゲーションや,人間に対する情報提供を正確に行う上で,重要なタスクである.
本稿では,与えられた複数の画像のそれぞれに対して,それらを互いに区別可能な自然言語ラベルを生成する問題を,画像情報ラベリングと呼ぶ.

画像情報を説明する自然言語テキストを生成する手法は,画像キャプション生成の分野で研究されている \cite{Farhadi2010,Vinyals2017,Dai2023}.
特に近年では,与えられた画像情報を説明する自然言語テキストの生成や,画像情報に関する質問に回答することのできる,マルチモーダルな事前学習済み大規模言語モデルが広く利用可能になっている \cite{Gan2022}.
深層学習モデルは,モデルの規模(パラメータ数)と訓練データの規模を増加させることで品質が向上し続ける傾向があり\cite{Devlin2019},画像キャプション生成においても有効性が示されている.
このことから,可能な限り大きなスケールで事前学習した共有の大規模言語モデルを,再学習することなく活用する技術の検討が重要になってきている.

しかし,大規模言語モデルは,大量のデータをそのまま入力することはできないことから,多数のデータの関係を考慮した処理には適していない.
このため,複数の画像に対して,それらを区別する自然言語ラベルを生成する,画像情報ラベリングの問題設定における性能は十分に検証されていない.
似た画像と区別できるようなキャプションを生成する手法の研究は行われている\cite{Vedantam2017,Cohn-Gordon2018,Nie2020,Andreas2016}が,これらの研究ではキャプション生成モデルの学習を伴うことから,大規模言語モデルの性能を活用することができないという課題がある.

本研究では,画像情報ラベリングタスクにおいて,大規模言語モデルを用いたテキスト生成手法の性能を評価すると共に,語用論的アプローチに基づいて性能の向上ができるかどうかを検討する.
語用論的アプローチは,聞き手の解釈を考慮して,誤解を避けるような言語を生成するアプローチであり,複数のテキスト生成タスクにおける品質の向上が報告されている.
本研究では,2つの手法を通して,画像情報ラベリングにおける語用論的アプローチの有効性を検証する.
1つ目の手法では,大規模言語モデルに生成させた画像情報のラベルの中から,誤解の起きやすいラベルを検出し,再び大規模言語モデルに修正させる.
2つ目の手法では,大規模言語モデルに直接ラベルを生成させるのではなく,多数の特徴をテキストで挙げさせて,そのうちのいくつかを選択してラベルを構成する.
この選択においては,誤解の起きやすさを評価する聞き手モデルを内部的に持つことによって,誤解可能性を最小化するような特徴の組み合わせを発見する最適化手法を提案する.

複数の部屋の画像が与えられたときに,それぞれの部屋を指し示すラベルを生成する実験を行った.
生成したラベルの評価にはAmazon Mechanical Turkを利用し,人間がどの程度誤解するかを検証した.
実験の結果から,テキスト長が短いラベルにおいて,提案する語用論的アプローチによって性能が改善することを示す.

% 本論文では,画像に対するラベル生成における新たなアプローチを提案する.具体的には,語用論的な観点から話し手モデルと聞き手モデルを導入し,その対話を通じてラベルの改善を図る.この新たなアプローチは,生成されたラベルが画像を一意に特定できることを保証することを目的としている.また,生成したラベルの評価にはAmazon Mechanical Turkを利用し,生成されたラベルが人間にとってどの程度理解しやすいかを検証する.本論文では,提案手法とその評価結果について詳細に述べる.

% 関連研究
\chapter{関連研究}

本章では,本研究で扱う画像情報ラベリングタスクに関連する研究として,画像に基づくテキスト生成技術について\ref{sec:relwork_image_captioning}節で述べる.
また,提案手法の着想に関連する研究として,語用論的アプローチに基づく機械学習手法について\ref{sec:relwork_pragmatic_ml}節で述べる.

\section{画像に基づくテキスト生成}
\label{sec:relwork_image_captioning}
画像内容に基づくテキスト生成技術の発展は,画像キャプション生成の研究によって牽引されてきた \cite{Farhadi2010}.
特に,深層学習を用いた画像キャプション生成では,ニューラルネットワークに画像の特徴を抽出し,得られた特徴に基づいて再帰型ニューラルネットワークでテキストを生成するモデルの枠組みが用いられる \cite{Vinyals2017}.
この枠組みにおいてXuら\cite{Xu2015}は,注意機構を用いて画像内の注目すべき箇所の情報を抽出することで,生成されるキャプションの品質が向上することを示した.

一方近年,タスクによらない画像やテキストの表現学習によって,深層学習の性能が大幅に向上することが明らかになってきた \cite{Devlin2019}.
この知見を発展させて,画像とテキストのマルチモーダルモデルを構成し,これを大規模な画像とテキストのデータセットを用いて特徴量表現を学習する手法が研究されている \cite{Yin2023,Gan2022}.
事前学習済みの大規模マルチモーダルモデルを用いることで,画像キャプション生成の品質が向上する\cite{Dai2023}だけでなく,画像のゼロショット分類\cite{Radford2021}や,画像質問応答\cite{Song2022}などのタスクの精度も向上することも示されている.

さらに近年では,モデル規模と訓練データをスケールアップすることで,テキストでの指示(プロンプト)に応じた画像に基づくテキスト生成ができることが明らかになった.
Liら\cite{Li2023}は,事前訓練済みの画像特徴量抽出器とテキスト生成器の間を橋渡しするQuerying Transformerを訓練する手法によって,画像に関する自然な会話や,柔軟な指示に応じた説明の生成ができることを示した.
GPT-4\cite{Bubeck2023}は,入力した画像を描画するプログラムを出力したり,グラフの解釈もできることを示している.
例えば,「この画像の特徴を10個挙げてください」というプロンプトと共に画像を入力することで,自然言語テキストで画像の特徴を得ることができる.

これらの深層学習に基づく手法はブラックボックスであり,どこまでのタスクを実現できるかは未知数であるが,基本的には1枚あるいは少数の画像とテキストの間を結びつける処理を想定している.
このため,多数の画像に対して,それらを区別する自然言語ラベルを生成する問題設定における性能は十分に検証されていない.
本研究では,画像情報ラベリングタスクにおいて,これらのテキスト生成手法の性能を評価すると共に,画像情報ラベリングタスクを分割して再構成することによるタスク性能の向上を検討する.

\section{語用論的アプローチに基づく機械学習}
\label{sec:relwork_pragmatic_ml}
言語の生成にあたって,聞き手の解釈を考慮して,誤解を避けるような語を選ぶことは,コミュニケーションにおいて重要な要素である.
このような,話し手による聞き手の解釈を考慮したアプローチは,計算語用論と呼ばれる\cite{Fried2023}.
Rational Speech Acts (RSA)フレームワークは,計算語用論の代表的な計算モデルである \cite{Frank2012,Goodman2016}.
RSAフレームワークでは,話し手と聞き手のモデルが,それぞれ相手の目的や解釈について再帰的に推論する過程を計算する.
例えば,RSAフレームワークにおける話し手モデルは,内部的に持つ聞き手モデルに発話を解釈させたときに,伝達したい意図通りの解釈が得られるような発話を選択する.

この知見を反映して,RSAフレームワークに基づいてテキスト生成タスクの品質を向上させる手法が研究されている\cite{Fried2017}.
Williamら\cite{Williams2015}は,室内のナビゲーションタスクにおいて,計算語用論に基づく手法を提案した.
話し手モデルによって生成された複数の指示を聞き手モデルに入力し,聞き手モデルが意図通りに解釈する指示を出力する.
SAILデータセットを用いた評価により,語用論的アプローチによってナビゲーションの成功率が向上することが示された.

画像キャプション生成においても,語用論的アプローチによって,聞き手が似た画像と区別できるようなキャプションを生成する手法が提案されている\cite{Vedantam2017,Cohn-Gordon2018,Nie2020}.
Andreasら\cite{Andreas2016}は,画像の参照ゲームにおいて,聞き手の誤解を低減するようなキャプションを生成する手法を提案した.
これらの手法は,ニューラルネットワークでモデル化された話し手と聞き手の学習を伴うことから,近年極めて高い品質を達成している大規模事前学習済みモデルを活用することが難しい.
本研究では,大規模事前学習済みモデルを再学習することなく用いる語用論的アプローチによって,誤解可能性を考慮した画像情報ラベリングを実現する手法を検討する.

\chapter{形式}

ここでは、論文の表紙および本体の記述方法について述べる。

\section{表紙}

表紙は、{\tt $\backslash$maketitle} によって作成するため、
以下の項目に相当する文字列をそれぞれ記述する。

\begin{description} \parskip=1pt
\item{題目: }
	題目は{\tt $\backslash$title} に記述する。
	行替えを行う場合は$\backslash\backslash$ を入力する。
	ただし、題目の最後に$\backslash\backslash$ を入力するとコンパイルが通らなくなるので注意する。
	なお、4行以上の題目の場合、表紙ページがあふれるためスタイルファイル``mast-jp-xxx.sty''を変更する必要がある
	(xxx は使用文字コードに合わせて euc, sjis, utf8 のいずれかになる)。
\item{著者名: }
	著者名は{\tt $\backslash$author} に記述する。
\item{指導教員名: }
	指導教教員は{\tt $\backslash$advisor} に記述する。
\item{年月: }
	年月は{\tt $\backslash$yearandmonth} に記述する。
	\\
	年月は別途指示された場合はそれにしたがう(指示がなければ提出時のものを記述する)。
\end{description}

\section{本体}

本体は1段組で記述する。

図表には番号と説明(caption)を付け、文章中で参照する。
表 \ref{table:fundamental_data_type}と図\ref{figure:sample} は
それぞれ表と図の例である。
表の説明は上に、図の説明は下に書くことが多い。
図の挿入に用いるパッケージについては使用環境に合わせて自由に選択してほしい。

\begin{table}[hbt]
\caption{表の例}
\label{table:fundamental_data_type}
\begin{center}
\begin{tabular}{| c | r | r | r | r |}
\hline
年 度 & 1年次 & 2年次 & 3年次 & 4年次 \\
\hline
1995 & 85 & 92 & 86 & 88 \\
1996 & 83 & 89 & 90 & 102 \\
1997 & 88 & 87 & 91 & 112 \\
1998 & 144 & 93 & 90 & 115 \\
\hline 
\end{tabular}
\end{center}
\end{table}
\medskip

\begin{figure}[htbp]
\begin{center}
\includegraphics[scale=0.6]{sample.eps}
%% epsfig.sty を使う場合には上の行を次のように変更する
% \psfig{file=sample.eps,scale=0.6}
\end{center}
\caption{図の例 (graphicx パッケージを使用)}
\label{figure:sample}
\end{figure}

詳しくは参考書\cite{okumura2010,yoshinaga2009}などを参照のこと。
奥村晴彦氏の「\TeX Wiki」 https://texwiki.texjp.org/ は
日本語の\TeX に関する情報が充実している。
具体的な文献の参照例として
本の例\cite{ware2004}、
雑誌論文の例\cite{meyer2009}、
予稿集の例\cite{hill2010}
を挙げておく。

\chapter*{謝辞}
\addcontentsline{toc}{chapter}{\numberline{}謝辞}	% 目次で左詰めするなら \numberline{} を削除する。

必須ではないが、書くことが望ましい。
\\
研究補助を受けている場合、他に指定がなければここに書く。

\newpage

\addcontentsline{toc}{chapter}{\numberline{}参考文献}	% 目次で左詰めするなら \numberline{} を削除する。
\renewcommand{\bibname}{参考文献}

%% 参考文献に jbibtex を使う場合
%\bibliographystyle{junsrt}
%\bibliography{samplebib}
%% [compile] jbibtex sample; platex sample; platex sample;
% 参考文献
\bibliographystyle{junsrt}
\bibliography{reference}
%% 参考文献を直接ファイルに含めて書く場合
% \begin{thebibliography}{99}
% \bibitem{okumura2010}
% 奥村晴彦: 
% LaTeX2e美文書作成入門 改訂第5版, 
% 技術評論社, 2010年.	% 「年」の字はなくてもよい。

% \bibitem{yoshinaga2009}
% 吉永徹美: 
% LaTeX2$\varepsilon$辞典, 
% 翔泳社, 2009年.		% 「年」の字はなくてもよい。

% \bibitem{ware2004} 
% Colin Ware:
% Information Visualization --- Perception for Design (Second Edition), 
% Morgan Kaufmann Publishers, 486~pp., 2004.
% %% 本のタイトルはイタリックにする流儀もある。その場合には全体を {\it ... } で囲む。
% %% 氏名の途中で改行させないためには、Colin~Ware のようにハードスペースを入れる。

% \bibitem{meyer2009}
% Miriah Meyer and Tamara Munzner:
% MizBee: A Multiscale Synteny Browser, 
% {\it IEEE Transactions on Visualization and Computer Graphics},
% Vol.~15, No.~6, pp.~897--904, 2009.

% \bibitem{hill2010}
% Emerson Murphy-Hill and Andrew P. Plack: 
% An Interactive Ambient Visualization for Code Smells, 
% {\it in Proceedings of the 2010 International Symposium on Software Visualization (SOFTVIS' 10)},
% pp.~5--14, 2010.

% \end{thebibliography}

\end{document}
